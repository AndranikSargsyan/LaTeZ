\setmainfont{DejaVu Sans}
\fontsize{12}{12}\selectfont

\noindent{\textbf{Թեզի վերնագիրը \`}}

\vspace{0.5cm}

\noindent{\textbf{Հայերենով\`} \thesisTitleArm,}

\vspace{0.5cm}

\noindent{\textbf{Ռուսերենով\`} \thesisTitleRus,}

\vspace{0.5cm}

\noindent{\textbf{Անգլերենով \`} \thesisTitleEng}

\vfill

\noindent{\textbf{Համառոտագիր}}

\vspace{0.5cm}

Այս աշխատանքը ուսումնասիրում է Երևանի Պետական Համալսարանի (ԵՊՀ) լայնածավալ դերը Հայաստանում ակադեմիական գերազանցության խթանման և սոցիալ-տնտեսական զարգացման խթանման գործում: Կիրառելով բազմաչափ հետազոտական մոտեցում, որը միավորում է ինչպես որակական, այնպես էլ քանակական մեթոդները, այս ուսումնասիրությունը տրամադրում է ԵՊՀ ազդեցության համապարփակ վերլուծություն կրթական, մշակութային և տնտեսական ոլորտներում:

Քանակական տվյալների հավաքագրման միջոցով՝ ներառյալ ավարտական ցուցանիշները, զբաղվածության արդյունքները և հետազոտական արդյունքները, և դասախոսների, ուսանողների, շրջանավարտների, տեղական ինքնակառավարման մարմինների պաշտոնյաների և բիզնեսի ղեկավարների հետ կիսակառույց հարցազրույցների որակական տվյալները՝ մենք խորանում ենք, թե ինչպես է ԵՊՀ-ն ձևավորում իր կրթական լանդշաֆտը և ազդում տնտեսության վրա։ Կրթական ազդեցության վերլուծությունը ընդգծում է ուսանողների կատարողականության ցուցանիշների զգալի բարելավումները՝ ուղղակիորեն փոխկապակցված ԵՊՀ-ի ընդլայնված ակադեմիական ծրագրերի և օժանդակ կառույցների հետ:

Հետազոտությունը եզրակացնում է, որ Երևանի պետական համալսարանը հայկական կրթության և զարգացման հիմնաքարն է, որի ազդեցությունն անցնում է ակադեմիական շրջանակներից դուրս՝ հասարակության հարստացման և տնտեսական ճկունության ոլորտներում: Աշխատանքը նպատակ ունի կրթության և քաղաքականության մշակման շահագրգիռ կողմերին արժեքավոր պատկերացումներ տրամադրել ազգային զարգացման մեջ բարձրագույն ուսումնական հաստատությունների ռազմավարական դերի վերաբերյալ:

\thispagestyle{empty}
\pagebreak
